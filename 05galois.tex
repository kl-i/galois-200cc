\documentclass[main.tex]{subfiles}

\begin{document}
\section{The Galois Correspondence}

\begin{thm} [\link{thm:char_gal}
  {Characterisation of Galois Extensions}]
  
  Let $(L,\io_L)$ be a $K$-extension.
  Then $(L,\io_L)$ is finite, normal, separable if and only if 
  $(L,\io_L)$ is Galois. 
  Furthermore, the Galois group of $(L,\io_L)$ is always $\aut_{K}{L}$
  and $[L : K] = |\aut_{K}{L}|$. 

\end{thm}
\begin{proof}
  $(\implies)$ We must give a finite subgroup $G \subseteq \aut_{}{L}$
  such that $\io_L K = L^G$. 
  The claim is that $G = \aut_{K}{L}$ works. 

  We first show that $\io_L K = L^G$. 
  It suffices $L^G \subseteq \io_L K$. 
  Let $a \in L^G$. It suffices that $\min(a,K)$ is linear. 
  Since $L$ is normal, $L$ splits $\min(a,K)$. 
  By separability of $\min(a,K)$,
  it suffices to show the only Galois $K$-conjugate of $a$ in $L$ is $a$.
  Let $\al \in L$ be a Galois $K$-conjugate of $a$. 
  It suffices to give $\phi_\al^L \in G = \aut_{K}{L}$ such that 
  $\phi_\al^L(a) = \al$.
  First by the \linkto{lem:char_simp}
  {characterisation of finite simple extensions}, 
  there exists $\phi_\al \in \emb{K}{(K(a),\io_L)}{(L,\io_L)}$
  such that $\phi_\al(a) = \al$. 
  Then by normality of $(L,\io_L)$ and 
  the \linkto{thm:embed}{embedding theorem for finite extensions},
  there exists $\phi_\al^L \in \emb{K(a)}{L}{(L,\phi_\al)}$.
  $\phi_\al^L$ is as desired.

  % Then by the \linkto{lem:char_simp}
  % {characterisation of finite simple extensions}, 
  % there exists $\phi_\al \in \emb{K}{(K(a),\io_L)}{(L,\io_L)}$
  % such that $\phi_\al(a) = \al$. 
  % Then by normality of $(L,\io_L)$ and 
  % the \linkto{thm:embed}{embedding theorem for finite extensions}, 
  % there exists $\phi_\al^L \in \emb{K(a)}{(L,\io_L)}{(L,\phi_\al)}$. 
  % We have $\phi_\al^L \circ \io_L = \phi_\al \circ \io_L = \io_L$,
  % i.e. $\phi_\al^L \in \aut_{K}{L} = G$. 
  % Thus $a = \phi_\al^L(a) = \phi_\al(a) = \al$. 
  % So $a$ only has one Galois $K$-conjugate in $(L,\io_L)$,
  % and since $\min(a,K)$ has no repeated roots in $(L,\io_L)$,
  % $\min(a,K)$ must be linear. 

  Finiteness of $G$ follows from 
  the \linkto{thm:embed}{embedding theorem for finite extensions}.
  \[ |G| = |\aut_{K}{L}| = |\emb{K}{L}{L}| \leq [L : K] \]

  $(\limplies)$ Let $G$ be a finite subgroup of $\aut_{}{L}$ such that 
  $\io_L K = L^G$. 
  WLOG we replace $K$ with $L^G$. 
  We first show that $L$ is algebraic, normal and separable over $L^G$. 
  Let $a \in L$. 
  Ideally, if $L$ is normal over $L^G$, then 
  it contains all Galois $L^G$-conjugates of $a$,
  which are precisely images of $a$ under $L^G$-automorphisms by
  the \linkto{lem:embed_simp}{embedding theorem for finite simple extensions}
  and \linkto{thm:embed}{finite extensions}.
  From this, we conjecture that
  \[ \min(a,K)(X) = \prod_{\al \in \Orb(a)} (X - \al) \]
  where $\Orb(a) := \set{\sigma(a) \st \sigma \in G}$.
  This suffices since the right exists, and is separable and split by $L$.
  % We define the \emph{orbit of $a$ under $G$} as the set 
  % $\Orb(a)$ of $\sigma(a)$ for $\sigma \in G$,
  % and the claim is that 
  % 
  % which is separable and split by $L$. 
  For the claim, it suffices that $\prod_{\al \in \Orb(a)} (X - \al)$
  is in $L^G[X]$ and irreducible. 
  The first follows from any $\sigma \in G$ permuting $\Orb(a)$.
  The latter follows since any $f \in L^G[X]$ that 
  has $a$ as a root must also have all of $\Orb(a)$ as roots via
  $0 = \sigma(ev_a f) = ev_{\sigma(a)}^\sigma f = ev_{\sigma(a)} f$
  for any $\sigma \in G$.

  % For the claim, we must show $\prod_{\al \in \Orb(a)} (X - \al)$
  % is in $L^G[X]$ and is irreducible. 
  % The first follows from 
  % $\sigma \prod_{\al \in \Orb(a)} (X - \al) 
  % = \prod_{\al^\prime \in \Orb(a)} (X - \al^\prime)$
  % for all $\sigma \in G$. 
  % For the latter, suppose $\prod_{\al \in \Orb(a)}(X - \al) = f(X)g(X)$
  % where $f, g \in L^G[X]$. 
  % Then since $a \in \Orb(a)$, $0 = ev_a(f g) = f(a) g(a)$
  % which implies either $f$ or $g$ has $a$ as a root. 
  % WLOG it's $f$. 
  % Then for all $\al \in \Orb(a)$, 
  % $\al = \sigma(a)$ for some $\sigma \in G$,
  % and thus $0 = \sigma(ev_a f) = ev_{\sigma(a)}^\sigma f 
  % = ev_{\sigma(a)} f = ev_\al f$.
  % This proves $\prod_{\al \in \Orb(a)} (X - \al) = f(X)$,
  % and hence the claim. 

  We now prove that $[L : L^G]$ is finite, and in fact, 
  $[L : L^G] \leq |G|$. 
  This requires one galaxy-brain idea,
  that $|G| = \dim_L (G \to L)$, 
  where $G \to L$ is the $L$-vectorspace of functions from $G$ to $L$. 
  It thus suffices to show that 
  for any $L^G$-linearly independent $X \subseteq L$, 
  there is an $L$-linearly independent $X_1 \subseteq G \to L$
  with $|X| = |X_1|$. 
  Let $X \subseteq L$ be $L^G$-linearly independent.
  Then for each $x \in X$, we have $ev_x : G \to L, \sigma \mapsto \sigma(x)$.
  Since $id \in G$, $\{ev_x\}_{x \in X}$ bijects with $X$. 
  We wish to show $\{ev_x\}_{x \in X}$ is $L$-linearly independent. 
  This is equivalent to showing for all finite $X_0 \subseteq X$,
  $\{ev_x\}_{x \in X_0}$ is $L$-linearly independent. 

  Let $X_0 \subseteq X$ be finite and $\sum_{x \in X_0} \la_x ev_x = 0$
  with $\la_x \in L$. 
  Suppose there exists $x_0 \in X_0$ such that $\la_{x_0} \neq 0$. 
  To get a contradiction, 
  it suffices to show for all $x \in X_0$, $\la_x \in L^G$,
  for then by evaluating at $id \in G$, 
  we have $0 = \sum_{x \in X_0} \la_x x$,
  implying all $\la_x = 0$. 
  So let $\sigma \in G$.
  By rescaling, WLOG $\la_{x_0} = 1$. 
  Then for all $\rho \in G$, 
  \[
    \sum_{x \in X_0 \setminus\{x_0\}} (\la_x - \sigma(\la_x)) ev_x (\rho)
    = \sum_{x \in X_0} \la_x ev_x (\rho)
      - \sigma \left( 
        \sum_{x \in X_0} \la_x ev_x(\sigma^{-1} \circ \rho) 
      \right)
    = 0 
  \]
  i.e. we have a $L$-linear combination of 
  $\{ev_x \,|\, x \in X_0 \setminus \{x_0\}\}$ yielding zero.
  By induction on the size of $X_0$, 
  $\la_x = \sigma(\la_x)$ for all $x \in X_0$.
  Thus for all $x \in X_0$, $\la_x \in L^G$ as desired. 
  
  The fact that $G = \aut_{L^G}{L}$ follows from 
  $|G| \leq |\aut_{L^G}{L}| = [L : L^G] \leq |G|$.
  This concludes the proof. 
\end{proof}

\begin{lem}[\link{lem:lift_normal}{Lifting Normality and Separability}]
  
  Let $(L,\io_L)$ be a $K$-extension, $E \subseteq L$ a field containing $\io_L K$
  so that $(E,\io_L)$ be naturally a $K$-extension. 
  Then the following are true.  
  \begin{enumerate}
    \item If $(L,\io_L)$ is a normal $K$-extension, 
    then $(L,\id{E})$ is a normal $E$-extension. 
    \item If $(L,\io_L)$ is a separable $K$-extension, 
    then $(L,\id{E})$ is a separable $E$-extension. 
  \end{enumerate}
\end{lem}
\begin{proof}~
  $(1)$ Easy via the first 
  \linkto{thm:char_normal}{characterisation of finite normal extensions}.
  % Let $(L,\io_L)$ be normal and $a \in L$. 
  % Then $\min(a,E) | \io_L \min(a,K)$ which splits in $L$ implies 
  % $\min(a,E)$ splits in $L$. 

  $(2)$ Let $(L,\io_L)$ be separable and $a \in L$. 
  Then $\min(a,E) | \io_L \min(a,K)$. 
  It suffices to show for all $g, f \in E[X]$, 
  $g | f$ and $f$ separable implies $g$ separable. 
  Let $g, f \in E[X]$, $g\dvd f$ and $f$ separable.
  Let $(M,\io_M)$ be an $E$-extension where $g$ splits. 
  By extending $(M,\io_M)$, WLOG $f$ splits in $M$ as well. 
  Then $f$ having no repeated roots implies $g$ has no repeated roots. 
\end{proof}

\begin{thm} [\link{thm:ftg2}
  {Fundamental Theorem of Galois Theory}] 

  Let $(L,\io_L)$ be a Galois $K$-extension,
  $I$ the partially ordered set of $K$-extensions inside $L$, 
  and $J$ the partially ordered set of subgroups of $\aut_{K}{L}$. 

  Then $\aut_{-}{L} : I \to J$ and $L^{-} : J \to I$ are surjections
  and hence bijections. 
  Furthermore, we have the following : 
  \begin{enumerate}
    \item (Degree equals Index) Let $E \in I$. 
    Then $[E : K] = [\aut_{K}{L} : \aut_{E}{L}]$.
    \item (Group Action) Let $E \in I$. 
    Then for all $\sigma \in \aut_{K}{L}$, 
    $\aut_{\sigma E}{L} = \sigma \aut_{E}{L} \sigma^{-1}$. 
    \item (Normality) Let $E \in I$. 
    Then $E$ is a normal $K$-extension if and only if 
    $\aut_{E}{L}$ is a normal subgroup of $\aut_{K}{L}$. 
    In this case, we have the isomorphism 
    $\aut_{K}{E} \cong \aut_{K}{L} / \aut_{E}{L}$. 
  \end{enumerate}
\end{thm}
\begin{proof}
  % Let $H \in J$. Then $L$ is a Galois $L^H$-extension, 
  % which by the characterisation of Galois extensions implies 
  % $\aut_{L^H}{L} = H$. 
  % So $\aut_{-}{L}$ is surjective. 
  % 
  % Let $E \in I$. Then by the characterisation of Galois extensions, 
  % $L$ is a finite, normal, separable $K$-extension, 
  % which implies $L$ is a finite, normal, separable $E$-extension, 
  % which by the characterisation again, implies 
  % $E = L^{\aut_{E}{L}}$. 
  % Hence, $L^{-}$ is surjective. 
  Surjectivity of $\aut_{-}{L}, L^-$ and $(1)$ follows from 
  \linkto{lem:lift_normal}{lifting normality and separability} and 
  the \linkto{thm:char_gal}{characterisation of Galois extensions}. 
  $(2)$ follows from definition. 

  $(3)$ First, $E$ normal $K$-extension implies $\aut_{E}{L}$ normal 
  follows from the \linkto{thm:char_norm}
  {characterisation of finite normal extensions} and $(2)$. 

  Now let $E \in I$ such that 
  $\aut_{E}{L}$ is a normal subgroup of $\aut_{K}{L}$.
  Then by $(2)$, for all $\sigma \in \aut_{K}{L}$, 
  $\sigma E = L^{\aut_{\sigma E}{L}} = L^{\sigma \aut_{E}{L} \sigma^{-1}}
  = L^{\aut_{E}{L}} = E$.
  We thus have a natural group morphism 
  $\aut_{K}{L} \to \aut_{}{E}, \sigma \mapsto \sigma$.
  Let $G$ be its image. 
  The kernel is clearly $\aut_{E}{L}$, 
  so by the first isomorphism theorem for groups, 
  we have $G \iso \aut_{K}{L} / \aut_{E}{L}$.
  Then $E^{G} = E \cap L^{\aut_{K}{L}} = E \cap \io_L K = \io_L K$,
  thus $E$ is a Galois $K$-extension, in particular normal. 
  We have thus also shown that $\aut_{K}{E} \cong \aut_{K}{L} / \aut_{E}{L}$. 
  % By the characterisation of Galois extensions, 
  % $L$ being Galois as a $K$-extension implies $L$ is Galois as a $E$-extension.
  % Thus $[E : K] = [L : K] / [L : E]
  % = |\aut_{K}{L}| / |\aut_{E}{L}| = [\aut_{K}{L} : \aut_{E}{L}]$. 
\end{proof}

\end{document}