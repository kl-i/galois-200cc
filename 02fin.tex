\documentclass[main.tex]{subfiles}

\begin{document}
\section{Finite Extensions and the Embedding Theorem}

\begin{dfn} [\link{dfn:fin_ext}
  {Extension Degree, Finite Extensions}] 
  
  Let $(L,\io_L)$ be a $K$-extension. 
  Then $L$ has a natural $K$-vectorspace structure with
  scalar multiplication as $k l := \io_L(k) l$ for $k \in K, l \in L$. 

  The \emph{degree of $(L,\io_L)$} is the dimension of $L$ as a $K$-vectorspace.
  We shall denote it with $[\io_L : K]$. 
  If the embedding of $K$ into $L$ is clear, then we write $[L : K]$. 
  We the degree of $(L,\io_L)$ is finite, 
  we call $(L,\io_L)$ a \emph{finite} extension. 

\end{dfn}

\begin{thm} [\link{thm:tower}
  {Tower Law of Extension Degree}]
  
  Let $(L,\io_L)$ be a $K$-extension and $(N,\io_N)$ a $L$-extension. 
  $(N,\io_N \circ \io_L)$ is then a $K$-extension. 

  Then $[\io_N \circ \io_L : K] = [\io_N : L] [\io_L : K]$. 
\end{thm} 
\begin{proof}
  Let $B_L \subseteq L$ be a $\io_L$-basis and 
  $B_N \subseteq N$ a $\io_N$-basis. 
  The claim is that $B_L B_N := \{ a b \,|\, a \in B_L, b \in B_N\}$
  is a $(\io_N \circ \io_L)$-basis of $N$ and has cardinality $B_L \times B_N$.

  (Cardinality) Let $(a_1, b_1), (a_2, b_2) \in B_L \times B_N$
  such that $ a_1 b_1 = a_2 b_2$. 
  This is then a non-trivial $L$-linear combination of elements in $B_N$,
  contradicting linear independence of $B_N$.
  The cardinality is thus as desired. 

  (Linear Independence) 
  Let $\sum_{(a, b) \in B_L \times B_N} \la_{a,b} a b = 0$
  where $\la_{a,b} \in K$ and only finitely many are non-zero. 
  Then we have 
  $\sum_{b \in B_N} \left(\sum_{a \in B_L} \la_{a,b} a\right) b = 0$,
  giving $\sum_{a \in B_L} \la_{a,b} a = 0$ by linear independence of $B_N$,
  which in turn gives $\la_{a,b} = 0$ by linear independence of $B_L$.
  
  (Spanning) 
  Let $x \in N$. 
  Since $B_N$ is spanning, we have $\sum_{b \in B_N} \la_b b = x$ 
  for some $\la_b \in L$, finitely many non-zero. 
  Then since $B_L$ is spanning, 
  we have $\sum_{a \in B_L} \mu_{a,b} a = \la_b$ for each $b \in N_B$, 
  where $\mu_{a,b} \in K$, finitely many non-zero. 
  So $\sum_{(a,b) \in B_L \times B_N} \mu_{a, b} a b = x$ as desired. 
\end{proof}

\begin{dfn} [\link{dfn:alg}
  {Extension generated by a Subset, 
  Evaluation Map, Algebraic Extension}]
  
  Let $(L,\io_L)$ be a $K$-extension and $A \subseteq L$. 
  Then the \emph{$K$-subextension generated by $A$} is defined as 
  the intersection of all fields in $L$ that contain $\io_L K \cup A$.
  This is denoted with $K(A)$.  
  If $A = \{a_1,\dots,a_{|A|}\}$ is finite, 
  then we just write $K(a_1,\dots,a_{|A|})$. 
  When $L = K(A)$, elements of $A$ are referred to as \emph{generators}. 
  $(L,\io_L)$ is called \emph{finitely generated} when 
  there exists finite $A \subseteq L$ such that $L = K(A)$. 

  Let $K[X], L[X]$ be the polynomial rings over $K$ and $L$. 
  Then $\io_L$ induces a ring morphism $K[X] \to L[X]$
  via $\sum_{n} f_n X^n \mapsto \sum_{n} \io_L(f_n) X^n$. 
  We will denote this map using $\io_L$ as well. 
  We can thus evaluate polynomials over $K$ at an element $a \in L$ 
  by the \emph{evaluation map} 
  $ev_a^{\io_L} : f \mapsto ev_a (\io_L f)$. 

  For $a \in L$, $ev_a^{\io_L}$ is a ring morphism from $K[X]$ to $L$. 
  Then for $f$ in the kernel of $ev_a^{\io_L}$, 
  we say \emph{$a$ is a root of $f$}. 
  If the kernel of $ev_a^{\io_L}$ contains non-zero polynomials, 
  then we say $a$ is \emph{algebraic over $K$}. 
  When all elements of $L$ are algebraic over $K$, 
  $(L,\io_L)$ is called an \emph{algebraic extension}.
\end{dfn}

\begin{lem} [\link{lem:char_fin_simp}
  {Characterisation of Finite Simple Extensions}]
  
  Let $(L,\io_L)$ be a $K$-extension and $a \in L$. 
  Then the following are equivalent : 
  \begin{enumerate}
    \item $(K(a),\io_L)$ is finite. 
    \item $(K(a),\io_L)$ is algebraic. 
    \item $a$ algebraic over $K$. 
  \end{enumerate}
\end{lem}
\begin{proof}
  $(2 \implies 3)$ clear. 

  $(3 \implies 1)$
  This follows from showing $K(a) = ev_a^{\io_L} K[X]$. 
  $K[X]$ is a PID so the kernel of $ev_a$ is generated by one element, 
  call it $\min(a,\io_L)$. 
  Since the image of $ev_a^{\io_L}$ is a ring inside $L$ 
  which is an integral domain, 
  it follows that $\min(a,\io_L)$ is prime. 
  Then since $K[X]$ is a PID, primes are irreducible 
  and quotienting by irreducibles give a field, 
  so $ev_a^{\io_L} K[X]$ is a field.
  It is clear that $ev_a K[X] \subseteq K(a)$ and hence it equal to it. 

  $(1 \implies 3)$ 
  If $K(a)$ is a finite dimensional $K$-vectorspace, 
  then the set $\{a^n\}_{n \in \N}$ must be linearly dependent. 
  This gives a polynomial $f \in K[X]$ such that $ev_a f = 0$. 

  $(1 \implies 2)$ Let $b \in K(a)$. 
  Then by the \linkto{thm:tower}{Tower Law}, 
  $[K(b) : K] \leq [K(a) : K]$, which is finite. 
  So by $(1 \implies 3)$, $b$ is algebraic over $K$. 
\end{proof}

\begin{thm} [\link{thm:char_fin}
  {Characterisation of Finite Extensions}]
  
  Let $(L,\io_L)$ be a $K$-extension.
  Then the following are equivalent : 
  \begin{enumerate}
    \item $(L,\io_L)$ is finite. 
    \item $(L,\io_L)$ is finitely generated and algebraic.   
    \item $(L,\io_L)$ is finitely generated and 
    the generators are algebraic over $K$. 
  \end{enumerate}
\end{thm}
\begin{proof}
  $(1 \implies 2)$ Let $A \subseteq L$ be a finite $K$-basis of $L$. 
  Then $L = K(A)$ and all elements of $L$ are algebraic by 
  the \linkto{lem:char_fin_simp}{characterisation of finite simple extensions}. 

  $(2 \implies 3)$ clear. 

  $(3 \limplies 1)$ Let $A$ finite $\subseteq L$ such that $L = K(A)$
  and all $a \in A$ are algebraic over $K$. 
  If $A$ is empty, $[L : K] = 1$. 
  So let $a \in A$. 
  Then by induction on the cardinality of $A$, 
  $(K(A\setminus\{a\}),\io_L)$ is a finite $K$-extension. 
  Since $a$ is algebraic over $K$, 
  it is certainly algebraic over $K(A\setminus\{a\})$. 
  So by the \linkto{lem:char_fin_simp}
  {characterisation of finite simple extensions}, 
  $L = K(A\setminus\{a\})(a)$ is a finite $K(A\setminus\{a\})$-extension. 
  Thus by the \linkto{thm:tower}{Tower Law}, 
  $[L : K] = [L : K(A\setminus\{a\})][K(A\setminus\{a\}) : K]$
  is finite. 
\end{proof}

\begin{dfn} [\link{dfn:gal_conj}
  {Minimal Polynomial, Galois Conjugates}] 
  
  Let $(L,\io_L)$ be a $K$-extension and $a \in L$ algebraic over $K$. 
  Then as in the proof of the \linkto{lem:char_fin_simp}
  {characterisation of finite simple extensions}, 
  the kernel of $ev_a^{\io_L} : K[X] \to L$ is generated by one element,
  the one of minimal degree and it is unique up to units. 
  However, there is a unique monic one. 
  It is defined as the \emph{minimal polynomial of $a$ over $K$},
  denoted $\min(a,\io_L)$. 
  If the embedding of $K$ into $L$ is clear, we write $\min(a,K)$. 

  Let $(M,\io_M)$ be another $K$-extensions and 
  $\al \in M$ also algebraic over $K$. 
  Then $a, \al$ are called \emph{Galois $K$-conjugates} when 
  $\min(a,K) = \min(\al, K)$. 
  It is not hard to check that being Galois $K$-conjugates is 
  an equivalence relation the ``set of elements in all $K$-extensions''. 
  \footnote{In case Bertrand Russell rises from his grave.}
\end{dfn}

\begin{lem} [\link{lem:embed_simp}
  {Embedding Theorem for Finite Simple Extensions}] 
  
  Let $(L,\io_L)$ be $K$-extensions and 
  $a \in L$ algebraic over $K$. 

  Then for all $K$-extensions $(N,\io_N)$, 
  the set of Galois $K$-conjugates of $a$ inside $N$ 
  bijects with $\emb{K}{K(a)}{N}$.
  In particular, $|\emb{K}{K(a)}{N}| \leq \deg \min(a,K) = [K(a) : K]$. 
\end{lem}
\begin{proof}
  Let $(N,\io_N)$ be a $K$-extension. 
  Given a $K$-embedding $\phi : (K(a),\io_L) \to (N,\io_N)$, 
  we have \[ 
    ev_{\phi(a)}^{\io_N} (\min(a,K)) 
    = ev_{\phi(a)}^{\phi \circ \io_L} (\min(a,K))
    = \phi\left( ev_a^{\io_L} \min(a,K) \right)
    = \phi(0) = 0
  \]
  i.e. $\min(a,K)$ has $\phi(a)$ as a root. 
  This shows $\phi(a)$ is not only algebraic, 
  but $\min(\phi(a),K)$ divides $\min(a,K)$ as well.
  $\min(a,K)$ is irreducible, so $\min(\phi(a),K) = \min(a,K)$. 
  Thus we have a map from $\emb{K}{K(a)}{N}$ to
  the set of Galois $K$-conjugates of $a$ in $N$ by $\phi \mapsto \phi(a)$. 
  Since $K(a)$ is generated by $a$, the above map is injective. 
  
  Now let $\al \in N$ be a Galois $K$-conjugate of $a$. 
  Then we have a chain of $K$-isomorphisms 
  \[ K(a) \cong K[X] / (\min(a,K)) 
  = K[X] / (\min(\al,K)) \cong K(\al) \]
  which gives a $K$-embedding of $\phi_\al : (K(a),\io_L) \to (N,\io_N)$. 
  Then $\phi_\al(a) = \al$ so our map is bijective. 
\end{proof}

\begin{lem} [\link{lem:partition}
  {Subextensions Partition Embeddings}]
  
  Let $(L,\io_L)$ an $K$-extension and 
  $E$ a field inside $L$ containing $\io_L K$. 
  So $(E,\io_L)$ is a $K$-extension and $L$ is a $E$-extension. 
  Let $(N,\io_N)$ be another $K$-extension. 

  Then we have the following bijection \begin{align*}
    \emb{K}{L}{N} \longleftrightarrow 
    \bigsqcup_{\io \in \emb{K}{E}{N}} \emb{E}{L}{\io}
  \end{align*}
  by sending $\io_0 \mapsto (\io_0,\io_0)$ and inversely  
  $(\io, \io_1) \mapsto \io_1$. 
  In particular when all these extensions are finite, 
  we obtain : 
  $|\emb{K}{L}{N}| = \sum_{\io \in \emb{K}{E}{N}} |\emb{E}{L}{\io}|$.  
\end{lem}
\begin{proof}
  Let $\io_0 : L \to N$ be a $K$-embedding. 
  Then $\io_0$ is naturally a $K$-embedding of $E$ to $N$
  and it is also an $E$-embedding of $L$ to $(N,\io_0)$.
  Conversely, let $\io : (E,\io_L) \to (N,\io_N)$ be a $K$-embedding
  and $\io_1 : L \to (N,\io)$ be a $E$-embedding. 
  Then since $\io_1 \circ \io_L = \io \circ \io_L = \io_N$, 
  $\io_1$ is a $K$-embedding of $L$ to $N$. 
  Thus the forward and inverse maps are well-defined.
  They are clearly inverses over each other so we have the result. 
\end{proof}

\begin{rmk}
  Despite its simplicity, the above lemma
  is \emph{essential} for proofs inducting on the degree of extensions.
\end{rmk}

\begin{thm} [\link{thm:embed}
  {Embedding Theorem for Finite Extensions}] 
  
  Let $(L,\io_L)$ be a finite $K$-extension.
  Then by the \linkto{thm:char_fin}{characterisation of finite extensions}, 
  we have a finite $A \subseteq L$ such that 
  $L = K(A)$ and all $a \in A$ are algebraic over $K$. 
  Let $(N,\io_N)$ be another $K$-extension, 
  and suppose for all $a \in A$, 
  $\min(a,K)$ has all its roots in $N$, 
  that is to say $\io_N \min(a,K)$ factorises 
  into linear polynomials in $N[X]$. 

  Then $0 < |\emb{K}{L}{N}| \leq [L : K]$
  and is equal when for all $a \in A$, 
  $\min(a,K)$ has \emph{no repeated} roots in $N$,
  i.e. $\io_N \min(a,K)$ has no repeated factors in $N[X]$.  

\end{thm} 
\begin{proof}
  If $A$ is empty, then the theorem is true.
  So let $a \in A$ and let $E = K(A \setminus \{a\})$. 
  Then by induction on the cardinality of $A$, 
  $0 < |\emb{K}{E}{N}| \leq [E : K]$
  and is equal when for all $a_1 \in A \setminus \{a\}$, 
  $\io_N \min(a_1,K)$ has no repeated roots. 
  Let $\io \in \emb{K}{E}{N}$. 
  Now $L = E(a)$.
  Since $a$ is a root of $\min(a,K)$,
  it is a root of $\io_L \min(a,K)$,
  so $\min(a,E)$ divides $\io_L \min(a,K)$. 
  Then since $\min(a,K)$ has all its roots in $N$,
  $\min(a,E)$ also has all its roots in $N$. 
  So by the \linkto{lem:char_fin_simp}
  {characterisation of finite simple extensions}, 
  we have $0 < |\emb{E}{L}{\io}|$,
  which in turn gives \[
    0 < |\emb{E}{L}{\io}| 
    \leq \sum_{\io \in \emb{K}{E}{N}} |\emb{E}{L}{\io}| = |\emb{K}{L}{N}|
  \]
  To complete the induction, 
  now suppose for all $a \in A$, $\min(a,K)$ has no repeated roots in $N$. 
  Then by induction, $|\emb{K}{E}{N}| = [E : K]$. 
  Furthermore, by the \linkto{lem:embed_simp}
  {embedding theorem for finite simple extensions}, 
  for all $\io \in \emb{K}{E}{N}$, 
  $\emb{E}{L}{\io}$ bijects with 
  the set of Galois $E$-conjugates of $a$ in $(N,\io)$.
  But since $\io_N \min(a, K)$ has no repeated factors 
  neither does $\io \min(a,E)$, 
  so it follows that the number of Galois $E$-conjugates of $a$ in $(N,\io)$
  equals the degree of $\min(a,E)$, which is equal to $[E(a) : E] = [L : E]$. 
  Thus the induction is complete by the tower law \[
    |\emb{K}{L}{N}| = \sum_{\io \in \emb{K}{E}{N}} |\emb{E}{L}{\io}|
    = |\emb{K}{E}{N}| [L : E] = [E : K] [L : E] = [L : K]
  \]
\end{proof}

\end{document}