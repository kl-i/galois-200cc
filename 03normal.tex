\documentclass[main.tex]{subfiles}

\begin{document}
\section{Normal Extensions}

\begin{dfn} [\link{dfn:normal}{Normal Extension}] 
  
  Let $(L,\io_L)$ be a $K$-extension and $f \in K[X]$. 
  Then we say \emph{$\io_L$ splits $f$} when 
  $\io_L f$ factorises into linear factors in $L[X]$. 
  If the embedding of $K$ into $L$ is clear, 
  we just say $L$ splits $f$. 

  Suppose $(L,\io_L)$ is algebraic. 
  Then it is called \emph{normal} when for all $a \in L$,  
  it contains all the Galois $K$-conjugates of $a$,
  i.e. $L$ splits $\min(a,K)$. 
\end{dfn}

\begin{thm} [\link{thm:split}
  {Splitting Polynomials}]
  
  Let $K$ be a field and $f \in K[X] \setminus K$.
  Then there exists a $K$-extension $(L,\io_L)$ such that
  $f$ has a root in $L$. 
  In particular, there exists a $K$-extension that splits $f$. 
\end{thm}
\begin{proof}
  Since $f$ is non-constant and $K[X]$ is a UFD, 
  there exists an irreducible $f_1$ that divides $f$.
  Let $L = K[X] / (f_1)$. 
  Then since $f_1$ is irreducible and $K[X]$ is a PID, 
  $L$ is a field and thus a $K$-extension. 
  Note that the image of the monomial $X$ in $L$ is a root of $f_1$,
  and hence a root of $f$.  
  To split $f$, use the above procedure to 
  inductively construct a desired extension.
\end{proof}

\begin{thm} [\link{thm:char_normal}
  {Characterisation of Finite Normal Extensions}]
  
  Let $(L,\io_L)$ be a finite $K$-extension.
  Then the following are equivalent : 
  \begin{enumerate}
    \item (Contains all Galois $K$-Conjugates) $(L,\io_L)$ normal. 
    \item (Contains all Galois $K$-Conjugates of Generators) 
    There exists $A \subseteq L$ a finite set of generators of $(L,\io_L)$
    such that for all $a \in A$, 
    $a$ is algebraic over $K$ and $L$ splits $\min(a,K)$. 
    \item (is a Splitting Field) 
    There exists a polynomial $f \in K[X]$ such that
    $L$ splits $f$ and is generated by the roots of $f$ in $L$.
    \item (Image Invariance) For all $K$-extensions $(N,\io_N)$
    and two $\io_0, \io_1 \in \emb{K}{L}{N}$, 
    $\io_0 L = \io_1 L$.
  \end{enumerate}
\end{thm}
\begin{proof} $(1 \implies 2 \implies 3)$ is clear. 

  $(3 \implies 4)$ The key is that roots of $f$ remain roots of $f$
  under $K$-extension morphisms. 

  Let $\io_L f(X) = \prod_{k = 1}^{\deg f} (X - a_k)$
  where $a_k \in L$.
  Then $(\io_0 \circ \io_L) f(X) = \prod_{k = 1}^{\deg f} (X - \io_0(a_k)) $. 
  Since $\io_0 \circ \io_L = \io_1 \circ \io_L$, 
  we have for every $a_l$ that \[
    0 = \io_1 (ev_{a_l}^{\io_L} f) 
    = ev_{\io_1(a_l)}^{\io_1 \circ \io_L} f
    = ev_{\io_1(a_l)}^{\io_0 \circ \io_L} f
    = \prod_{k = 1}^{\deg f} (\io_1(a_l) - \io_0(a_k))
  \] 
  Hence for all $a_l$, 
  there exists $a_k$ such that $\io_1(a_l) = \io_0(a_k)$.
  Since $L = K(a_1,\dots,a_{\deg f})$, 
  this shows that $\io_1 L \subseteq \io_0 L$.
  By symmetrical argument, $\io_0 L \subseteq \io_1 L$ as well. 

  $(4 \implies 1)$ Let $a \in L$. 
  Since $(L,\io_L)$ is finite, $\min(a,K)$ exists.
  We do not know if $L$ splits $\min(a,K)$,
  but there exists an $L$-extension $(M,\io_M)$ such that 
  $M$ splits $\min(a,K)$.
  We seek to show that all Galois $K$-conjugates of $a$ in $M$ are in $\io_M L$.
  So let $\al \in M$ be a Galois $K$-conjugate of $a$. 
  We have the following situation. 
  \begin{figure} [H]
    \centering
    \begin{tikzcd}
      K \arrow["\io_L",r] & 
      K(a) \arrow["\subseteq",r] \arrow["\phi_\al",rd] & 
      L \arrow["\io_M",d]\\
      & & M
    \end{tikzcd}
  \end{figure}
  By the \linkto{lem:embed_simp}{embedding theorem for finite simple extensions}, 
  there exists $\phi_\al \in \emb{K}{K(a)}{M}$
  that maps $a \mapsto \al$. 
  Suppose we have an $\io_1 \in \emb{K(a)}{L}{\phi_\al}$.
  Then certainly $\io_1 \in \emb{K}{L}{\io_M \circ \io_L}$. 
  Also, trivially $\io_M \in \emb{K}{L}{\io_M \circ \io_L}$. 
  So $\io_1 L = \io_M L$ implies $\al \in \io_M L$ as desired. 
  It thus suffices to give an $\io_1 \in \emb{K(a)}{L}{\phi_\al}$. 
  Well, since $(L,\io_L)$ is finite, it is also a finite $K(a)$-extension, 
  so it is generated by some finite subset $B$
  whose elements are all algebraic over $K(a)$. 
  Then we can extend $M$ so that it splits all $\min(b,K(a))$ for $b \in B$. 
  Thus by the \linkto{thm:embed}{embedding theorem}, 
  we have an $\io_1 \in \emb{K(a)}{L}{\phi_\al}$. 
\end{proof}


\end{document}