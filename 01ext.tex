\documentclass[main.tex]{subfiles}

\begin{document}
\section{Field Extensions and the Essence of Galois Theory}

\begin{dfn} [\link{dfn:field_morphisms}
  {Field Morphisms, Isomorphisms, Automorphisms}]
  % Let $K$ be a commutative ring. 
  % Then $K$ is a \emph{field} when the following are satisfied : 
  % \begin{enumerate}
  %   \item (Non-Trivial) $1 \neq 0 \in K$.
  %   \item (Non-Zero implies Unit) 
  %   $\forall\, x \in K \setminus \{0\}, \exists\, y \in K, x y = 1$.  
  % \end{enumerate}
  % Since non-zero elements of a field are multiplicatively invertible, 
  % they are called \emph{units}. 

  A \emph{field morphism} $\io : K \to L$ between fields is
  just a ring morphism, that is, a function satisfying the following : 
  \begin{enumerate}
    \item (Additive)
    $\forall\, x, y \in K, \io(x + y) = \io(x) + \io(y)$. 
    \item (Multiplicative) 
    $\forall\, x, y \in K, \io(x y) = \io(x) \io(y)$. 
    \item (Preserves One) $\io(1) = 1$.  
  \end{enumerate} 
  A bijective field morphism is called an \emph{isomorphism}. 
  In particular, 
  isomorphisms of a field to itself are called \emph{automorphisms}. 
  We denote the set of all field automorphisms of a field $L$ 
  with $\aut L$. 
\end{dfn}

\begin{thm}[\link{thm:field_emb}{Fields Embed into other Fields}]
  
  Let $\io : K \to L$ be a morphism of fields. 

  Then $\io$ is injective. 
  In particular, $\io$ is a field isomorphism from $K$ to its image. 
\end{thm}
\begin{proof}
  Injectivity of $\io$ is equivalent to $\io^{-1}(0) = \{0\}$. 
  Let $x \in \io^{-1}(0)$.
  If it is non-zero, then there exists $y \in K$ such that $x y = 1$. 
  Then $1 = \io(1) = \io(x) \io(y) = 0$
  which is a contradiction, since $L$ is non-trivial. 
\end{proof}

\begin{dfn} [\link{dfn:ext}
  {Extensions of a Field, Extension Embeddings, Isomorphisms, 
  Automorphisms}] 
  
  Let $K$ be a field. 
  Then a \emph{$K$-extension} is a pair $(L,\io_L)$
  where $L$ is a field and $\io_L : K \to L$ is a field morphism. 
  We say \emph{$L$ extends $K$}. 

  A \emph{$K$-embedding} between 
  two $K$-extensions $(L,\io_L), (N,\io_N)$ is 
  a field morphism $\io : L \to N$ such that $\io \circ \io_L = \io_N$.
  We say $K$ is \emph{preserved} in such a map.
  If a $K$-embedding is bijective, 
  it is called a \emph{$K$-isomorphism}. 
  In particular, 
  $K$-isomorphisms from a $K$-extension to itself are 
  called \emph{$K$-automorphisms}. 

  The set of $K$-embeddings between $(L,\io_L), (N,\io_N)$ is 
  denoted with $\emb{K}{\io_L}{\io_N}$. 
  If the field morphisms $\io_L, \io_N$ are clear, 
  we write $\emb{K}{L}{N}$. 
  In a similar manner, 
  we denote the set of $K$-automorphisms of $(L,\io_L)$ with 
  $\aut_K(\io_L)$, or simply $\aut_K L$ when $\io_L$ is clear. 

\end{dfn}

\begin{dfn}[\link{dfn:fixed_galois}
  {Fixed Subfields, Galois Extensions}]
  
  Let $L$ be a field. 
  Then $\aut L$ forms a group under function composition. 
  Let $G$ be a finite subgroup of $\aut L$.
  We say \emph{$G$ acts on $L$}. 

  Then the \emph{subfield fixed by $G$}, denoted $L^G$, 
  is the field of elements in $a \in L$ such that 
  for all $\sigma \in G$, $\sigma (a) = a$. 
  Its inclusion into $L$ is a field morphism, 
  making $L$ into an $L^G$-extension. 

  Let $K$ be a field with $(L,\io_L)$ as a $K$-extension. 
  Then $(L,\io_L)$ is called a \emph{Galois} when 
  $\io_L K = L^G$ for some finite subgroup $G \subseteq \aut L$. 
  In this case, $G$ is called a \emph{Galois group of $(L,\io_L)$}. 
\end{dfn}

\begin{dfn} [\link{dfn:gc}{Antitone Galois Connection}]

  Let $(I,\leq), (J, \leq)$ be partially ordered sets. 
  Then an \emph{antitone Galois connection between $I$ and $J$} is 
  a pair of maps $F : I \to J$ and $G : J \to I$ such that
  \begin{enumerate}
    \item (Antitonicity) 
    For all $i_0 \leq i_1 \in I$ and $j_0 \leq j_1 \in J$, 
    we have $F(i_1) \leq F(i_0)$ and $G(j_1) \leq G(j_0)$. 
    \item (Adjunction) For all $i \in I, j \in J$, 
    we have $i \leq G(j) \iff j \leq F(i)$. 
  \end{enumerate}
\end{dfn}

\begin{thm} [\link{thm:gc_bij}
  {Antitone Galois Connections gives Bijection on Images}]

  Let $(I,\leq), (J,\leq)$ be partially ordered sets and 
  $F : I \to J, G : J \to I$ an antitone Galois connection.

  Then $G \circ F \circ G = G$ and $F \circ G \circ F = F$, 
  that is to say $F : G(J) \to F(I)$ and $G : F(I) \to G(J)$ are bijections. 
\end{thm}
\begin{proof}
  Straight forward. 
\end{proof}

\begin{cor} [\link{cor:gc_galois}
  {The Antitone Galois Connection of Galois Theory}] 
  
  Let $(L,\io_L)$ be a $K$-extension. 
  Let $I$ be the set of fields inside $L$ containing $\io_L K$, 
  i.e. the set of $K$-extensions inside $L$. 
  This is partially ordered by inclusion. 
  Let $J$ be the set of subgroups of 
  the group of $K$-automorphisms of $L$, $\aut_K L$. 
  This is also partially ordered by inclusion. 

  Then the maps \begin{alignat*}{3}
    \aut_{-} L : I \longrightarrow J &&\quad L^- : J \longrightarrow I \\
    E \mapsto \aut_E L &&\quad H \mapsto L^H
  \end{alignat*}
  form an antitone Galois connection and are bijective on their images. 
\end{cor}

\begin{thm} [\link{thm:ftg1}
  {Fundamental Theorem of Galois Theory}] 

  Let $(L,\io_L)$ be a $K$-extension,
  $I$ the partially ordered set of $K$-extensions inside $L$, 
  and $J$ the partially ordered set of subgroups of $\aut_K L$. 
  Let $(L,\io_L)$ be Galois. 

  Then $\aut_{-} L : I \to J$ and $L^{-} : J \to I$ are surjections. 
\end{thm}

\begin{rmk}
  The \linkto{thm:ftg2}{full fundamental theorem} has more subtleties, 
  but the essence of Galois theory is to characterise Galois extensions 
  in order to exploit the bijection between subextensions and subgroups. 
\end{rmk}

\end{document}